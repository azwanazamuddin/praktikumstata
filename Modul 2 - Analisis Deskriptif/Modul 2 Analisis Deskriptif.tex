% Options for packages loaded elsewhere
\PassOptionsToPackage{unicode}{hyperref}
\PassOptionsToPackage{hyphens}{url}
%
\documentclass[
]{article}
\usepackage{amsmath,amssymb}
\usepackage{lmodern}
\usepackage{iftex}
\ifPDFTeX
  \usepackage[T1]{fontenc}
  \usepackage[utf8]{inputenc}
  \usepackage{textcomp} % provide euro and other symbols
\else % if luatex or xetex
  \usepackage{unicode-math}
  \defaultfontfeatures{Scale=MatchLowercase}
  \defaultfontfeatures[\rmfamily]{Ligatures=TeX,Scale=1}
\fi
% Use upquote if available, for straight quotes in verbatim environments
\IfFileExists{upquote.sty}{\usepackage{upquote}}{}
\IfFileExists{microtype.sty}{% use microtype if available
  \usepackage[]{microtype}
  \UseMicrotypeSet[protrusion]{basicmath} % disable protrusion for tt fonts
}{}
\makeatletter
\@ifundefined{KOMAClassName}{% if non-KOMA class
  \IfFileExists{parskip.sty}{%
    \usepackage{parskip}
  }{% else
    \setlength{\parindent}{0pt}
    \setlength{\parskip}{6pt plus 2pt minus 1pt}}
}{% if KOMA class
  \KOMAoptions{parskip=half}}
\makeatother
\usepackage{xcolor}
\IfFileExists{xurl.sty}{\usepackage{xurl}}{} % add URL line breaks if available
\IfFileExists{bookmark.sty}{\usepackage{bookmark}}{\usepackage{hyperref}}
\hypersetup{
  hidelinks,
  pdfcreator={LaTeX via pandoc}}
\urlstyle{same} % disable monospaced font for URLs
\usepackage{graphicx}
\makeatletter
\def\maxwidth{\ifdim\Gin@nat@width>\linewidth\linewidth\else\Gin@nat@width\fi}
\def\maxheight{\ifdim\Gin@nat@height>\textheight\textheight\else\Gin@nat@height\fi}
\makeatother
% Scale images if necessary, so that they will not overflow the page
% margins by default, and it is still possible to overwrite the defaults
% using explicit options in \includegraphics[width, height, ...]{}
\setkeys{Gin}{width=\maxwidth,height=\maxheight,keepaspectratio}
% Set default figure placement to htbp
\makeatletter
\def\fps@figure{htbp}
\makeatother
\usepackage[normalem]{ulem}
% Avoid problems with \sout in headers with hyperref
\pdfstringdefDisableCommands{\renewcommand{\sout}{}}
\setlength{\emergencystretch}{3em} % prevent overfull lines
\providecommand{\tightlist}{%
  \setlength{\itemsep}{0pt}\setlength{\parskip}{0pt}}
\setcounter{secnumdepth}{-\maxdimen} % remove section numbering
\ifLuaTeX
  \usepackage{selnolig}  % disable illegal ligatures
\fi

\author{}
\date{}

\begin{document}

\hypertarget{modul-ii-analisis-deskriptif}{%
\section{\texorpdfstring{MODUL II\\
ANALISIS DESKRIPTIF
}{MODUL II ANALISIS DESKRIPTIF }}\label{modul-ii-analisis-deskriptif}}

\hypertarget{tujuan}{%
\subsection{Tujuan}\label{tujuan}}

\begin{enumerate}
\def\labelenumi{\arabic{enumi}.}
\item
  \begin{quote}
  Praktikan mampu memahami analisis deskriptif
  \end{quote}
\item
  \begin{quote}
  Praktikan mampu melakukan analisis deskriptif
  \end{quote}
\end{enumerate}

\hypertarget{section}{%
\subsection{}\label{section}}

\hypertarget{alat-dan-data}{%
\subsection{Alat dan Data}\label{alat-dan-data}}

Alat:

\begin{enumerate}
\def\labelenumi{\arabic{enumi}.}
\item
  \begin{quote}
  Komputer dan Perangkatnya
  \end{quote}
\item
  \begin{quote}
  Aplikasi Stata
  \end{quote}
\end{enumerate}

Data:

\begin{enumerate}
\def\labelenumi{\arabic{enumi}.}
\item
  \begin{quote}
  Proposal Penelitian (Sebagai Acuan Praktikum)
  \end{quote}
\item
  \begin{quote}
  Perangkat Survei
  \end{quote}
\item
  \begin{quote}
  Data Hasil Survei
  \end{quote}
\end{enumerate}

\hypertarget{teori-dasar}{%
\subsection{Teori Dasar}\label{teori-dasar}}

Secara umum, statistik deskriptif bertujuan untuk mereduksi data
sehingga dengan penampilan yang lebih sederhana dapat mendeskripsikan
karakteristik data.

\hypertarget{analisis-deskriptif}{%
\subsection{\texorpdfstring{\uline{ANALISIS
DESKRIPTIF}}{ANALISIS DESKRIPTIF}}\label{analisis-deskriptif}}

Analisis deskriptif adalah suatu cara menggambarkan persoalan
berdasarkan data yang dimiliki, yakni dengan cara menata data tersebut
sedemikian rupa sehingga dengan mudah dapat dipahami karakteristiknya,
dapat dijelaskan dan digunakan untuk keperluan selanjutnya melalui
sebuah proses reduksi data. Reduksi data merupakan metode untuk
meringkas sekumpulan data ke dalam kumpulan data yang lebih kecil yang
menggambarkan pengamatan awal tanpa mengorbankan informasi penting.
Analisis statistik deskriptif lebih berhubungan dengan kegiatan
pengumpulan, peringkasan, dan penyajian data. Analisis ini melibatkan
penggunaan sejumlah kecil angka, tabel, dan grafik untuk menyimpulkan
sederet angka yang lebih besar. Secara umum, metode reduksi data dibagi
menjadi tiga, yaitu:

\begin{enumerate}
\def\labelenumi{\Alph{enumi}.}
\item
  \begin{quote}
  Reduksi Data Dasar
  \end{quote}
\end{enumerate}

\begin{quote}
Reduksi data dasar membantu menunjukkan distribusi data secara
menyeluruh dan cepat. Reduksi data dasar terbagi menjadi tiga jenis,
yaitu:
\end{quote}

\begin{enumerate}
\def\labelenumi{\arabic{enumi}.}
\item
  \begin{quote}
  Reduksi Data dalam Nilai Baku
  \end{quote}
\end{enumerate}

\begin{quote}
Reduksi data dengan cara membuat perbandingan antara jumlah kejadian
dalam suatu kategori terhadap jumlah kejadian pada kategori lain maupun
keseluruhan, baik yang nyata maupun yang potensial terjadi. Bentuk
reduksi data dalam nilai baku adalah proporsi, prosentase, \emph{rates},
dan rasio.
\end{quote}

\begin{enumerate}
\def\labelenumi{\arabic{enumi}.}
\setcounter{enumi}{1}
\item
  \begin{quote}
  Reduksi Data dalam Tabel
  \end{quote}
\end{enumerate}

\begin{quote}
Reduksi data dalam bentuk tabel distribusi frekuensi, yaitu tabel yang
meringkas distribusi nilai variabel dengan menunjukkan jumlah kejadian
pada setiap kategori nilai variabel.
\end{quote}

\begin{enumerate}
\def\labelenumi{\arabic{enumi}.}
\setcounter{enumi}{2}
\item
  \begin{quote}
  Reduksi Data dalam Chart/Grafik
  \end{quote}
\end{enumerate}

\begin{quote}
Reduksi data dalam Chart/Grafik, yaitu reduksi data dalam bentuk gambar
yang dapat menunjukkan chart/grafik yang bermacam-macam sesuai
kebutuhan. Namun, chart/grafik yang paling umum digunakan adalah
\emph{pie chart} yang menggunakan permukaan lingkaran untuk
menggambarkan distribusi proporsi antar kategori, \emph{bar chart} yang
menggunakan balok untuk menggambarkan distribusi proporsi, histogram
yang menggunakan balok untuk menggambarkan distribusi frekuensi data
interval rasio, dan \emph{frequency polygons} yang menggunakan titik
untuk menggambarkan distribusi frekuensi data interval rasio.
\end{quote}

\begin{enumerate}
\def\labelenumi{\Alph{enumi}.}
\setcounter{enumi}{1}
\item
  \begin{quote}
  Ukuran Kecenderungan Memusat
  \end{quote}
\end{enumerate}

\begin{quote}
Ukuran kecenderungan memusat merupakan ukuran yang menunjukkan pada
nilai variabel mana kejadian cenderung terkonsentrasi. Ukuran
kecenderungan memusat yang digunakan ada tiga yaitu:
\end{quote}

\begin{enumerate}
\def\labelenumi{\arabic{enumi}.}
\item
  \begin{quote}
  Modus
  \end{quote}
\end{enumerate}

\begin{quote}
Modus adalah nilai yang paling banyak muncul. Modus digunakan sebagai
tipe skala pengukuran kecenderungan memusat pada data nominal.
\end{quote}

\begin{enumerate}
\def\labelenumi{\arabic{enumi}.}
\setcounter{enumi}{1}
\item
  \begin{quote}
  Median
  \end{quote}
\end{enumerate}

\begin{quote}
Median adalah nilai variabel dari objek yang mempunyai setengah jumlah
objek di atasnya dan setengah jumlah objek di bawahnya setelah semua
nilai variabel objek diurutkan dari yang terkecil sampai yang terbesar.
Median digunakan sebagai tipe skala pengukuran kecenderungan memusat
pada data ordinal.
\end{quote}

\begin{enumerate}
\def\labelenumi{\arabic{enumi}.}
\setcounter{enumi}{2}
\item
  \begin{quote}
  Mean
  \end{quote}
\end{enumerate}

\begin{quote}
Mean adalah rata-rata hitung nilai variabel yan g dimiliki seluruh
objek. Mean digunakan sebagai tipe skala pengukuran kecenderungan
memusat pada data interval rasio. Adapun rumus mean yaitu:
\end{quote}

\(= \ \frac{X_{i}}{n}\ \)

\begin{enumerate}
\def\labelenumi{\Alph{enumi}.}
\setcounter{enumi}{2}
\item
  \begin{quote}
  Ukuran Persebaran Data
  \end{quote}
\end{enumerate}

\begin{quote}
Ukuran persebaran data adalah suatu ukuran yang memberikan suatu
indikasi tingkat heterogenitas atau keragaman dalam distribusi nilai
variabel. Terdapat tiga skala pengukuran yang digunakan sebagai ukuran
persebaran data yaitu:
\end{quote}

\begin{enumerate}
\def\labelenumi{\arabic{enumi}.}
\item
  \begin{quote}
  Indeks Variansi Kualitatif
  \end{quote}
\end{enumerate}

\begin{quote}
Indeks variansi kualitatif adalah rasio jumlah variasi yang diamati
secara nyata dalam suatu distribusi nilai pada variasi maksimum yang
dapat terjadi dalam distribusi tersebut. IQV kerap digunakan untuk
melihat persebaran data pada data nominal. Indeks variansi kualitatif
(IQV) dapat dihitung menggunakan rumus:
\end{quote}

\(IQV = \ \frac{k\ (N^{2} - \ f_{i}^{2})}{N^{2\ }(k - 1)}\)

\begin{quote}
Dengan:

k = jumlah kategori

N = jumlah kasus/objek

f\textsubscript{i} = jumlah kasus/objek dalam kategori i

IQV = 0 memiliki arti bahwa tingkat heterogenitas rendah atau dapat
dikatakan seragam

IQV = 1 memiliki arti bahwa tingkat heterogenitas tinggi
\end{quote}

\begin{enumerate}
\def\labelenumi{\arabic{enumi}.}
\setcounter{enumi}{1}
\item
  \begin{quote}
  Rentang
  \end{quote}
\end{enumerate}

\begin{quote}
Rentang merupakan selisih nilai variabel objek yang terbesar dan yang
terkecil. Rentang digunakan untuk melihat persebaran data pada data
ordinal.
\end{quote}

\begin{enumerate}
\def\labelenumi{\arabic{enumi}.}
\setcounter{enumi}{2}
\item
  \begin{quote}
  Variansi dan Standar Deviasi/Simpangan Baku
  \end{quote}
\end{enumerate}

\begin{quote}
Variansi adalah nilai rata-rata kuadrat selisih nilai variabel setiap
objek dengan nilai rata-rata variabel. Variansi dan standar deviasi
digunakan untuk melihat persebaran data pada interval rasio. Nilai
standar deviasi terkecil adalah 0 dimana semakin besar nilai variansi
dan standar deviasi maka data makin bervariasi. Variansi dan standar
deviasi dapat dihitung dengan rumus:
\end{quote}

\(\text{Variansi\ }\left( S^{2} \right) = \ \frac{(X_{i} - \ )}{n}\)

\(\text{Standar\ Deviasi\ }(S) = \ \)

\begin{quote}
Dengan:

N = jumlah data

X\textsubscript{i} = data ke-i

X̄ = rata-rata sampel/populasi
\end{quote}

\hypertarget{pengolahan-data-dan-analisis}{%
\subsection{Pengolahan Data dan
Analisis}\label{pengolahan-data-dan-analisis}}

Dalam praktikum ini akan dilakukan uji missing value, uji normalitas
data, analisis deskriptif (reduksi data).

\hypertarget{uji-missing-value}{%
\subsubsection{UJI MISSING VALUE}\label{uji-missing-value}}

Pada uji ini digunakan data ``Dataset Analisis Deskriptif''.

\begin{enumerate}
\def\labelenumi{\arabic{enumi}.}
\item
  Mencari missing value
\end{enumerate}

\begin{quote}
Missing value pada variabel yang dituju dapat ditemukan dengan command:

\textbf{list Nama\_Responden Total\_Pendapatan if missing(
Total\_Pendapatan )}

Dengan command tersebut akan ditampilkan variabel Nama\_Responden yang
variabel Total\_Pendapatan-nya kosong dalam bentuk tabel seperti
berikut.

\includegraphics[width=2.54167in,height=2.23958in]{media/image7.png}
\end{quote}

\begin{enumerate}
\def\labelenumi{\arabic{enumi}.}
\setcounter{enumi}{1}
\item
  Menghilangkan data \emph{missing value}
\end{enumerate}

\begin{quote}
Apabila terdapat data missing value, maka dapat dilakukan penghilangan
data objek tersebut. Pada data nominal, opsi ini lah yang lebih tepat
digunakan. Penghilangan data dapat menggunakan command:

\textbf{drop if missing(Alamat)}

\includegraphics[width=2.17708in,height=0.35417in]{media/image5.png}
\end{quote}

\begin{enumerate}
\def\labelenumi{\arabic{enumi}.}
\setcounter{enumi}{2}
\item
  Mengganti missing value dengan mean
\end{enumerate}

\begin{quote}
Pada data interval rasio, ukuran pemusatan data yang digunakan adalah
mean sehingga untuk missing value pada skala data interval rasio dapat
diganti dengan mean menggunakan command:

\textbf{replace Total\_Pendapatan = (mean) if
missing(Total\_Pendapatan)}

Mean diisi dengan mean variabel tersebut yang dapat didapatkan
menggunakan command: \textbf{summarize Total\_Pendapatan}

Pada praktikum ini, mean dari variabel Total\_Pendapatan adalah 4092841
sehingga setelah dimasukkan command untuk mengganti missing value dengan
mean didapatkan hasil seperti berikut pada STATA.

\includegraphics[width=9.97128in,height=0.53836in]{media/image6.png}
\end{quote}

\hypertarget{analisis-deskriptif-1}{%
\subsubsection{ANALISIS DESKRIPTIF}\label{analisis-deskriptif-1}}

\uline{Reduksi Data: Tabel Frekuensi}

Command: \textbf{tabulate Jenis\_Kelamin}

Dengan memasukkan command tersebut akan muncul tabel frekuensi dari
variabel Jenis\_Kelamin sebagai berikut.

\includegraphics[width=3.625in,height=1.52083in]{media/image2.png}

\uline{Reduksi Data: Grafik}

\begin{enumerate}
\def\labelenumi{\arabic{enumi}.}
\item
  Pie Chart
\end{enumerate}

\begin{quote}
Command: \textbf{graph pie, over(Jenis\_Kelamin) plabel(\_all percent)}

Dengan command tersebut akan muncul pie chart yang memuat variabel
Jenis\_Kelamin seperti berikut.

\includegraphics[width=4.30719in,height=3.15536in]{media/image8.png}
\end{quote}

\begin{enumerate}
\def\labelenumi{\arabic{enumi}.}
\setcounter{enumi}{1}
\item
  Bar Chart
\end{enumerate}

\begin{quote}
Command: \textbf{graph bar (count), over(Anggota\_Keluarga)}

Dengan command tersebut akan muncul bar chart yang memuat variabel
Anggota\_Keluarga seperti berikut.

\includegraphics[width=3.61269in,height=2.64625in]{media/image1.png}
\end{quote}

\begin{enumerate}
\def\labelenumi{\arabic{enumi}.}
\setcounter{enumi}{2}
\item
  Histogram
\end{enumerate}

\begin{quote}
Command: \textbf{histogram Usia}

Dengan command tersebut akan muncul histogram yang memuat variabel Usia
seperti berikut.

\includegraphics[width=4.01953in,height=2.9418in]{media/image3.png}
\end{quote}

\uline{Ukuran Kecenderungan Terpusat dan Tingkat Persebaran}

Command: \textbf{summarize Total\_Pendapatan}

Dengan command tersebut akan muncul tabel berikut.

\includegraphics[width=10.66236in,height=0.95512in]{media/image11.png}

Command: \textbf{summarize Total\_Pendapatan, detail}

Dengan command tersebut akan muncul tabel berikut. Tabel dengan command
ini lebih lengkap dengan tambahan nilai median (50\%), variansi,
skewness, dan kurtosis.

\includegraphics[width=11.98865in,height=3.66172in]{media/image13.png}

\uline{Tabel Deskripsi Variabel Yang Diklasifikasikan Berdasarkan Konten
Variabel Lain}

Command: \textbf{table Jenis\_Kelamin, contents(count Frekuensi\_Sakit
min Frekuensi\_Sakit max Frekuensi\_Sakit mean Frekuensi\_Sakit )}

\includegraphics[width=10.84565in,height=1.36953in]{media/image10.png}

Untuk melihat box-plot dari Frekuensi\_Sakit berdasarkan Jenis\_Kelamin
dapat digunakan command: \textbf{graph box Frekuensi\_Sakit,
over(Jenis\_Kelamin)}

\includegraphics[width=4.64501in,height=3.39957in]{media/image4.png}

\uline{Tabel Deskripsi Variabel Yang Diklasifikasikan Berdasarkan Dua
Konten Variabel Lain}

Command: \textbf{table Jenis\_Kelamin, contents(count Frekuensi\_Sakit
min Frekuensi\_Sakit max Frekuensi\_Sakit mean Frekuensi\_Sakit )
by(Asuransi\_Kesehatan)}

\includegraphics[width=10.14791in,height=2.56285in]{media/image12.png}

Untuk melihat box-plot dari Frekuensi\_Sakit berdasarkan Jenis\_Kelamin
dan Asuransi\_Kesehatan dapat digunakan command\textbf{: graph box
Frekuensi\_Sakit, over(Jenis\_Kelamin) over( Asuransi\_Kesehatan)}

\includegraphics[width=4.58834in,height=3.35779in]{media/image9.png}

\hypertarget{pustaka}{%
\subsection{Pustaka}\label{pustaka}}

Sawitri, Dewi. Maryati, Sri. 2014. \emph{Metode Analisis Perencanaan.}
Penerbit Universitas Terbuka: Tangerang.

Healey J F, Statistics, A Tool for Social Research, Ninth Edition,
Wadsworth Publishing Company, 2012.

Kachigan, Sam Kash. 1982. \emph{Statistical Analysis}. Radius Press: New
York.

Trihendradi, C. 2008. \emph{Step by Step STATA 16 Analisis Data
Statistik}. Penerbit ANDI: Yogyakarta

\end{document}
